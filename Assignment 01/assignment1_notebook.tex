% Options for packages loaded elsewhere
\PassOptionsToPackage{unicode}{hyperref}
\PassOptionsToPackage{hyphens}{url}
%
\documentclass[
]{article}
\usepackage{lmodern}
\usepackage{amsmath}
\usepackage{ifxetex,ifluatex}
\ifnum 0\ifxetex 1\fi\ifluatex 1\fi=0 % if pdftex
  \usepackage[T1]{fontenc}
  \usepackage[utf8]{inputenc}
  \usepackage{textcomp} % provide euro and other symbols
  \usepackage{amssymb}
\else % if luatex or xetex
  \usepackage{unicode-math}
  \defaultfontfeatures{Scale=MatchLowercase}
  \defaultfontfeatures[\rmfamily]{Ligatures=TeX,Scale=1}
\fi
% Use upquote if available, for straight quotes in verbatim environments
\IfFileExists{upquote.sty}{\usepackage{upquote}}{}
\IfFileExists{microtype.sty}{% use microtype if available
  \usepackage[]{microtype}
  \UseMicrotypeSet[protrusion]{basicmath} % disable protrusion for tt fonts
}{}
\makeatletter
\@ifundefined{KOMAClassName}{% if non-KOMA class
  \IfFileExists{parskip.sty}{%
    \usepackage{parskip}
  }{% else
    \setlength{\parindent}{0pt}
    \setlength{\parskip}{6pt plus 2pt minus 1pt}}
}{% if KOMA class
  \KOMAoptions{parskip=half}}
\makeatother
\usepackage{xcolor}
\IfFileExists{xurl.sty}{\usepackage{xurl}}{} % add URL line breaks if available
\IfFileExists{bookmark.sty}{\usepackage{bookmark}}{\usepackage{hyperref}}
\hypersetup{
  pdftitle={Assignment 1},
  pdfauthor={Chiebuka Onwuzurike},
  hidelinks,
  pdfcreator={LaTeX via pandoc}}
\urlstyle{same} % disable monospaced font for URLs
\usepackage[margin=1in]{geometry}
\usepackage{color}
\usepackage{fancyvrb}
\newcommand{\VerbBar}{|}
\newcommand{\VERB}{\Verb[commandchars=\\\{\}]}
\DefineVerbatimEnvironment{Highlighting}{Verbatim}{commandchars=\\\{\}}
% Add ',fontsize=\small' for more characters per line
\usepackage{framed}
\definecolor{shadecolor}{RGB}{248,248,248}
\newenvironment{Shaded}{\begin{snugshade}}{\end{snugshade}}
\newcommand{\AlertTok}[1]{\textcolor[rgb]{0.94,0.16,0.16}{#1}}
\newcommand{\AnnotationTok}[1]{\textcolor[rgb]{0.56,0.35,0.01}{\textbf{\textit{#1}}}}
\newcommand{\AttributeTok}[1]{\textcolor[rgb]{0.77,0.63,0.00}{#1}}
\newcommand{\BaseNTok}[1]{\textcolor[rgb]{0.00,0.00,0.81}{#1}}
\newcommand{\BuiltInTok}[1]{#1}
\newcommand{\CharTok}[1]{\textcolor[rgb]{0.31,0.60,0.02}{#1}}
\newcommand{\CommentTok}[1]{\textcolor[rgb]{0.56,0.35,0.01}{\textit{#1}}}
\newcommand{\CommentVarTok}[1]{\textcolor[rgb]{0.56,0.35,0.01}{\textbf{\textit{#1}}}}
\newcommand{\ConstantTok}[1]{\textcolor[rgb]{0.00,0.00,0.00}{#1}}
\newcommand{\ControlFlowTok}[1]{\textcolor[rgb]{0.13,0.29,0.53}{\textbf{#1}}}
\newcommand{\DataTypeTok}[1]{\textcolor[rgb]{0.13,0.29,0.53}{#1}}
\newcommand{\DecValTok}[1]{\textcolor[rgb]{0.00,0.00,0.81}{#1}}
\newcommand{\DocumentationTok}[1]{\textcolor[rgb]{0.56,0.35,0.01}{\textbf{\textit{#1}}}}
\newcommand{\ErrorTok}[1]{\textcolor[rgb]{0.64,0.00,0.00}{\textbf{#1}}}
\newcommand{\ExtensionTok}[1]{#1}
\newcommand{\FloatTok}[1]{\textcolor[rgb]{0.00,0.00,0.81}{#1}}
\newcommand{\FunctionTok}[1]{\textcolor[rgb]{0.00,0.00,0.00}{#1}}
\newcommand{\ImportTok}[1]{#1}
\newcommand{\InformationTok}[1]{\textcolor[rgb]{0.56,0.35,0.01}{\textbf{\textit{#1}}}}
\newcommand{\KeywordTok}[1]{\textcolor[rgb]{0.13,0.29,0.53}{\textbf{#1}}}
\newcommand{\NormalTok}[1]{#1}
\newcommand{\OperatorTok}[1]{\textcolor[rgb]{0.81,0.36,0.00}{\textbf{#1}}}
\newcommand{\OtherTok}[1]{\textcolor[rgb]{0.56,0.35,0.01}{#1}}
\newcommand{\PreprocessorTok}[1]{\textcolor[rgb]{0.56,0.35,0.01}{\textit{#1}}}
\newcommand{\RegionMarkerTok}[1]{#1}
\newcommand{\SpecialCharTok}[1]{\textcolor[rgb]{0.00,0.00,0.00}{#1}}
\newcommand{\SpecialStringTok}[1]{\textcolor[rgb]{0.31,0.60,0.02}{#1}}
\newcommand{\StringTok}[1]{\textcolor[rgb]{0.31,0.60,0.02}{#1}}
\newcommand{\VariableTok}[1]{\textcolor[rgb]{0.00,0.00,0.00}{#1}}
\newcommand{\VerbatimStringTok}[1]{\textcolor[rgb]{0.31,0.60,0.02}{#1}}
\newcommand{\WarningTok}[1]{\textcolor[rgb]{0.56,0.35,0.01}{\textbf{\textit{#1}}}}
\usepackage{graphicx}
\makeatletter
\def\maxwidth{\ifdim\Gin@nat@width>\linewidth\linewidth\else\Gin@nat@width\fi}
\def\maxheight{\ifdim\Gin@nat@height>\textheight\textheight\else\Gin@nat@height\fi}
\makeatother
% Scale images if necessary, so that they will not overflow the page
% margins by default, and it is still possible to overwrite the defaults
% using explicit options in \includegraphics[width, height, ...]{}
\setkeys{Gin}{width=\maxwidth,height=\maxheight,keepaspectratio}
% Set default figure placement to htbp
\makeatletter
\def\fps@figure{htbp}
\makeatother
\setlength{\emergencystretch}{3em} % prevent overfull lines
\providecommand{\tightlist}{%
  \setlength{\itemsep}{0pt}\setlength{\parskip}{0pt}}
\setcounter{secnumdepth}{-\maxdimen} % remove section numbering
\ifluatex
  \usepackage{selnolig}  % disable illegal ligatures
\fi

\title{Assignment 1}
\author{Chiebuka Onwuzurike}
\date{}

\begin{document}
\maketitle

\hypertarget{instructions}{%
\subsubsection{Instructions:}\label{instructions}}

You can use it to create a document that consists of your answers to the
questions and your R code. Please change your name in the above text
from ``Your Name Here'' to your name.

\hypertarget{about-r-markdown-notebooks}{%
\subsubsection{About R Markdown
Notebooks}\label{about-r-markdown-notebooks}}

This is an \href{http://rmarkdown.rstudio.com}{R Markdown} Notebook.
When you execute code within the notebook, the results appear beneath
the code. It can also be used to generate PDFs of your results. Add a
new chunk by clicking the \emph{Insert Chunk} button on the toolbar or
by pressing \emph{Cmd+Option+I}. The preview shows you a rendered HTML
copy of the contents of the editor. Consequently, unlike \emph{Knit},
\emph{Preview} does not run any R code chunks. Instead, the output of
the chunk when it was last run in the editor is displayed.

\hypertarget{r-markdown-instruction}{%
\subsubsection{R markdown instruction}\label{r-markdown-instruction}}

In R markdown files, we put code into `chunks', like the one below. You
can click the `run' buttom on the code chunk below to run it. As you can
see, the code chunk returns the output of the R code.

\begin{Shaded}
\begin{Highlighting}[]
\NormalTok{variable\_1 }\OtherTok{\textless{}{-}} \DecValTok{1}
\NormalTok{variable\_2 }\OtherTok{\textless{}{-}} \DecValTok{2}
\NormalTok{variable\_1 }\SpecialCharTok{+}\NormalTok{ variable\_2}
\end{Highlighting}
\end{Shaded}

\begin{verbatim}
## [1] 3
\end{verbatim}

\hypertarget{problem-1}{%
\subsubsection{Problem 1:}\label{problem-1}}

\hypertarget{section}{%
\paragraph{1.1}\label{section}}

Our first task will be to read the dataset called `class\_data.csv'. To
do so, we will use the function `fread' from the package `data.table'.
Please run the chunk below to load the library and read the data. Run
this chunk:

\begin{Shaded}
\begin{Highlighting}[]
\CommentTok{\# Load library \textquotesingle{}data.table\textquotesingle{}}
\FunctionTok{library}\NormalTok{(data.table)}
\FunctionTok{library}\NormalTok{(purrr)}
\end{Highlighting}
\end{Shaded}

\begin{verbatim}
## 
## Attaching package: 'purrr'
\end{verbatim}

\begin{verbatim}
## The following object is masked from 'package:data.table':
## 
##     transpose
\end{verbatim}

\begin{Shaded}
\begin{Highlighting}[]
\CommentTok{\# Read the data \#}
\NormalTok{class\_data }\OtherTok{\textless{}{-}} \FunctionTok{fread}\NormalTok{(}\StringTok{\textquotesingle{}student\_list\_msba2021spring.csv\textquotesingle{}}\NormalTok{)}
\end{Highlighting}
\end{Shaded}

To check that the data has been read successfully, you can type the name
of the data structure (class\_data) into the console. Or look at the
environment tab in Rstudio.

Let's print the first few lines using the `head' command.

\begin{Shaded}
\begin{Highlighting}[]
\FunctionTok{head}\NormalTok{(class\_data)}
\end{Highlighting}
\end{Shaded}

\begin{verbatim}
##              name  user_id           email treatment_group section
## 1: Patel, Manushi manuship manuship@bu.edu               2      B1
## 2:      Yu, Lequn    lequn    lequn@bu.edu               4      B1
## 3:   Sun, Zhiyuan jacobszy jacobszy@bu.edu               1      B1
## 4:         Li, Bo boli0315 boli0315@bu.edu               0      B1
## 5:  Leng, Linghan   lenglh   lenglh@bu.edu               3      B1
## 6:       Guo, Kai kaiguo96 kaiguo96@bu.edu               0      B1
\end{verbatim}

\hypertarget{reading-your-name}{%
\paragraph{1.2 Reading your name}\label{reading-your-name}}

Our next goal is to find the row associated with your name. Each row has
a number, and we can isolate it by that number. We can also specify
which columns show up. The code below gets the first row, for example.
Modify it so that your name shows up (use the columns for name and
user\_id).

\begin{Shaded}
\begin{Highlighting}[]
\NormalTok{class\_data[}\DecValTok{30}\NormalTok{, }\FunctionTok{list}\NormalTok{(name, user\_id)]}
\end{Highlighting}
\end{Shaded}

\begin{verbatim}
##              name  user_id
## 1: Veytsman, Ryan veytsman
\end{verbatim}

We can also reference a row by the value of that row. The code below
isolates the rows for which the college is QST. Modify it so that it
finds the row whose first\_name is your name.

\begin{Shaded}
\begin{Highlighting}[]
\NormalTok{class\_data[section }\SpecialCharTok{==} \StringTok{\textquotesingle{}A1\textquotesingle{}}\NormalTok{]}
\end{Highlighting}
\end{Shaded}

\begin{verbatim}
##                           name  user_id           email treatment_group section
##  1:                 Zhu, Shuyi shuyizhu shuyizhu@bu.edu               0      A1
##  2:                Wang, Jiaqi jwang311 jwang311@bu.edu               3      A1
##  3:                Zhang, Ying   yingyz   yingyz@bu.edu               2      A1
##  4:                Li, Mingwei mingweil mingweil@bu.edu               1      A1
##  5:              Chen, Chuning chuningc chuningc@bu.edu               1      A1
##  6:               Li, Jiazheng jli12320 jli12320@bu.edu               0      A1
##  7:                   Ming, Yi   yming2   yming2@bu.edu               3      A1
##  8:              Shi, Kangjing   kangjs   kangjs@bu.edu               2      A1
##  9:              Fowlkes, Leah fowlkesl fowlkesl@bu.edu               3      A1
## 10:                   Shi, Man   mshi10   mshi10@bu.edu               2      A1
## 11:               Lu, Jingjing ljjjjjjj ljjjjjjj@bu.edu               3      A1
## 12:                Huang, Anyi  ayhuang  ayhuang@bu.edu               0      A1
## 13:                 Xu, Jiahui jiahuixu jiahuixu@bu.edu               0      A1
## 14:           Torres, Maraline matorres matorres@bu.edu               0      A1
## 15:               Nagesh, Jaya  jnagesh  jnagesh@bu.edu               1      A1
## 16:            Demacker, Paula demacker demacker@bu.edu               0      A1
## 17:              Wu, Tsung Yen     tywu     tywu@bu.edu               4      A1
## 18:          Lensing, Michelle mlensing mlensing@bu.edu               2      A1
## 19:            Chen, Chang-Han chchen31 chchen31@bu.edu               0      A1
## 20:             Mao, Tianzheng  markmao  markmao@bu.edu               3      A1
## 21:            Zhang, Yuanming    mm121    mm121@bu.edu               4      A1
## 22:                 Cao, Yujia   yc0908   yc0908@bu.edu               0      A1
## 23:             Kalwe, Shamika  shamika  shamika@bu.edu               3      A1
## 24:             Ling, Jingjing   jjling   jjling@bu.edu               1      A1
## 25:                 Shan, Ziqi     ziqi     ziqi@bu.edu               1      A1
## 26:        Lawrence, Christian chrislaw chrislaw@bu.edu               1      A1
## 27:               Feng, Yuchen   ycfeng   ycfeng@bu.edu               2      A1
## 28:            Hou, Chang-Hung    chhou    chhou@bu.edu               0      A1
## 29:              Grose, Jordan  jpgrose  jpgrose@bu.edu               1      A1
## 30:            Guo, Zhangcheng     kguo     kguo@bu.edu               3      A1
## 31:                 Sun, Shiqi  shiqi17  shiqi17@bu.edu               0      A1
## 32:                 Huo, Lulin huolulin huolulin@bu.edu               1      A1
## 33:               Jiang, Alice  ayjiang  ayjiang@bu.edu               0      A1
## 34: Sentissi El Idrissi, Selma selmasen selmasen@bu.edu               1      A1
## 35:              Zhang, Weilin wzhang97 wzhang97@bu.edu               1      A1
## 36:                Chen, Cheng  chengcc  chengcc@bu.edu               1      A1
## 37:              Liang, Xuanqi   xuanqi   xuanqi@bu.edu               2      A1
## 38:                   Lv, Feng lvfeng00 lvfeng00@bu.edu               4      A1
## 39:               Gong, Yuting wgongyty wgongyty@bu.edu               1      A1
## 40:               Wang, Yantao yantwang yantwang@bu.edu               1      A1
##                           name  user_id           email treatment_group section
\end{verbatim}

\hypertarget{section-1}{%
\subsubsection{1.3}\label{section-1}}

Find the treatment group associated with your name.

\begin{Shaded}
\begin{Highlighting}[]
\NormalTok{class\_data[name }\SpecialCharTok{==} \StringTok{\textquotesingle{}Onwuzurike, Chiebuka\textquotesingle{}}\NormalTok{,}\FunctionTok{list}\NormalTok{(name, treatment\_group)]}
\end{Highlighting}
\end{Shaded}

\begin{verbatim}
##                    name treatment_group
## 1: Onwuzurike, Chiebuka               3
\end{verbatim}

\hypertarget{section-2}{%
\subsubsection{1.4}\label{section-2}}

Create a new column called `section\_code' that takes the value of 1 if
the student is in section `A1' and 0 otherwise. The purpose is to use
this column to calculate what share of students is in each section.

\begin{Shaded}
\begin{Highlighting}[]
\DocumentationTok{\#\#\# How to create a new column:}
\CommentTok{\# class\_data[, test\_column1 := 1]}
\CommentTok{\# class\_data$test\_column2 \textless{}{-} 2}
\CommentTok{\# Verify that these two columns were created}
\CommentTok{\# class\_data[, list(test\_column1, test\_column2)]}
\NormalTok{class\_data[, section\_code }\SpecialCharTok{:}\ErrorTok{=} \ConstantTok{NULL}\NormalTok{]}
\end{Highlighting}
\end{Shaded}

\begin{verbatim}
## Warning in `[.data.table`(class_data, , `:=`(section_code, NULL)): Column
## 'section_code' does not exist to remove
\end{verbatim}

To create a conditional column we can use the `ifelse' function. The
ifelse function has three parts: a. The first part determines the
condition. b. The part after the first comma determines what happens if
a) is true. c.~The part after the second comma determines what happens
if b) is true. Let's try this! The code below create a column that takes
the value 1 if your user\_id starts with the letter w and 0 otherwise.
Note, we use the `substr' function to get the first letter.

\begin{Shaded}
\begin{Highlighting}[]
\NormalTok{class\_data[, starts\_with\_w }\SpecialCharTok{:}\ErrorTok{=} \FunctionTok{ifelse}\NormalTok{(}\FunctionTok{substr}\NormalTok{(user\_id, }\DecValTok{1}\NormalTok{, }\DecValTok{1}\NormalTok{)  }\SpecialCharTok{==} \StringTok{\textquotesingle{}w\textquotesingle{}}\NormalTok{, }\DecValTok{1}\NormalTok{, }\DecValTok{0}\NormalTok{)]}
\CommentTok{\# Check that it works:}
\NormalTok{class\_data[}\DecValTok{1}\SpecialCharTok{:}\DecValTok{6}\NormalTok{, }\FunctionTok{list}\NormalTok{(starts\_with\_w, user\_id)]}
\end{Highlighting}
\end{Shaded}

\begin{verbatim}
##    starts_with_w  user_id
## 1:             0 manuship
## 2:             0    lequn
## 3:             0 jacobszy
## 4:             0 boli0315
## 5:             0   lenglh
## 6:             0 kaiguo96
\end{verbatim}

Create a new column called `section\_code' that takes the value of 1 if
the student is in section `A1' and 0 otherwise.

\begin{Shaded}
\begin{Highlighting}[]
\NormalTok{class\_data[, section\_code }\SpecialCharTok{:}\ErrorTok{=} \FunctionTok{ifelse}\NormalTok{(section }\SpecialCharTok{==} \StringTok{"A1"}\NormalTok{,}\DecValTok{1}\NormalTok{,}\DecValTok{0}\NormalTok{)]}
\NormalTok{class\_data}
\end{Highlighting}
\end{Shaded}

\begin{verbatim}
##                                name  user_id           email treatment_group
##  1:                  Patel, Manushi manuship manuship@bu.edu               2
##  2:                       Yu, Lequn    lequn    lequn@bu.edu               4
##  3:                    Sun, Zhiyuan jacobszy jacobszy@bu.edu               1
##  4:                          Li, Bo boli0315 boli0315@bu.edu               0
##  5:                   Leng, Linghan   lenglh   lenglh@bu.edu               3
##  6:                        Guo, Kai kaiguo96 kaiguo96@bu.edu               0
##  7: Moral Cevallos, Antonio Gabriel  antogmc  antogmc@bu.edu               4
##  8:                    Gong, Yulong   yulong   yulong@bu.edu               4
##  9:                      Tang, Qiqi qiqitang qiqitang@bu.edu               2
## 10:                      Xie, Muyan    xiemy    xiemy@bu.edu               0
## 11:               Mankiewich, Peter   pmmank   pmmank@bu.edu               3
## 12:                    Pan, Chenzhi panchenz panchenz@bu.edu               3
## 13:                  Jafari, Mohsen  roozbeh  roozbeh@bu.edu               0
## 14:                    Zhang, Yichi   yichiz   yichiz@bu.edu               1
## 15:                    Wang, Yixuan   yx1201   yx1201@bu.edu               2
## 16:                    Guo, Jiajian prosylar prosylar@bu.edu               3
## 17:                  Huang, Tzu-Hua huahuang huahuang@bu.edu               0
## 18:                 Demiralp, Yigit   yigitd   yigitd@bu.edu               1
## 19:                  Zhou, Yangyang  yyzhous  yyzhous@bu.edu               1
## 20:                      Li, Zuoyin   zli179   zli179@bu.edu               0
## 21:          Venkateswaran, Ruchika ruchikav ruchikav@bu.edu               1
## 22:           Derhovanessians, Maro   marodh   marodh@bu.edu               1
## 23:                     Qiu, Chenli  karlanq  karlanq@bu.edu               0
## 24:            Onwuzurike, Chiebuka chiebuka chiebuka@bu.edu               3
## 25:                      Kim, Bosoo  bwk5100  bwk5100@bu.edu               2
## 26:                        Wu, Ying  amberwy  amberwy@bu.edu               2
## 27:                  Wang, Yi-Shuan phoenixw phoenixw@bu.edu               2
## 28:                    Zheng, Yuzhe  yzzheng  yzzheng@bu.edu               2
## 29:                     Zhou, Xiang  xzxiang  xzxiang@bu.edu               4
## 30:                  Veytsman, Ryan veytsman veytsman@bu.edu               2
## 31:                    Moradi, Tiam  tmoradi  tmoradi@bu.edu               2
## 32:                     McCoy, John  sjmccoy  sjmccoy@bu.edu               4
## 33:                     Yu, Xinping     xpyu     xpyu@bu.edu               4
## 34:                  Leung, Jeffrey jeleung2 jeleung2@bu.edu               2
## 35:                      Zhu, Shuyi shuyizhu shuyizhu@bu.edu               0
## 36:                     Wang, Jiaqi jwang311 jwang311@bu.edu               3
## 37:                     Zhang, Ying   yingyz   yingyz@bu.edu               2
## 38:                     Li, Mingwei mingweil mingweil@bu.edu               1
## 39:                   Chen, Chuning chuningc chuningc@bu.edu               1
## 40:                    Li, Jiazheng jli12320 jli12320@bu.edu               0
## 41:                        Ming, Yi   yming2   yming2@bu.edu               3
## 42:                   Shi, Kangjing   kangjs   kangjs@bu.edu               2
## 43:                   Fowlkes, Leah fowlkesl fowlkesl@bu.edu               3
## 44:                        Shi, Man   mshi10   mshi10@bu.edu               2
## 45:                    Lu, Jingjing ljjjjjjj ljjjjjjj@bu.edu               3
## 46:                     Huang, Anyi  ayhuang  ayhuang@bu.edu               0
## 47:                      Xu, Jiahui jiahuixu jiahuixu@bu.edu               0
## 48:                Torres, Maraline matorres matorres@bu.edu               0
## 49:                    Nagesh, Jaya  jnagesh  jnagesh@bu.edu               1
## 50:                 Demacker, Paula demacker demacker@bu.edu               0
## 51:                   Wu, Tsung Yen     tywu     tywu@bu.edu               4
## 52:               Lensing, Michelle mlensing mlensing@bu.edu               2
## 53:                 Chen, Chang-Han chchen31 chchen31@bu.edu               0
## 54:                  Mao, Tianzheng  markmao  markmao@bu.edu               3
## 55:                 Zhang, Yuanming    mm121    mm121@bu.edu               4
## 56:                      Cao, Yujia   yc0908   yc0908@bu.edu               0
## 57:                  Kalwe, Shamika  shamika  shamika@bu.edu               3
## 58:                  Ling, Jingjing   jjling   jjling@bu.edu               1
## 59:                      Shan, Ziqi     ziqi     ziqi@bu.edu               1
## 60:             Lawrence, Christian chrislaw chrislaw@bu.edu               1
## 61:                    Feng, Yuchen   ycfeng   ycfeng@bu.edu               2
## 62:                 Hou, Chang-Hung    chhou    chhou@bu.edu               0
## 63:                   Grose, Jordan  jpgrose  jpgrose@bu.edu               1
## 64:                 Guo, Zhangcheng     kguo     kguo@bu.edu               3
## 65:                      Sun, Shiqi  shiqi17  shiqi17@bu.edu               0
## 66:                      Huo, Lulin huolulin huolulin@bu.edu               1
## 67:                    Jiang, Alice  ayjiang  ayjiang@bu.edu               0
## 68:      Sentissi El Idrissi, Selma selmasen selmasen@bu.edu               1
## 69:                   Zhang, Weilin wzhang97 wzhang97@bu.edu               1
## 70:                     Chen, Cheng  chengcc  chengcc@bu.edu               1
## 71:                   Liang, Xuanqi   xuanqi   xuanqi@bu.edu               2
## 72:                        Lv, Feng lvfeng00 lvfeng00@bu.edu               4
## 73:                    Gong, Yuting wgongyty wgongyty@bu.edu               1
## 74:                    Wang, Yantao yantwang yantwang@bu.edu               1
##                                name  user_id           email treatment_group
##     section starts_with_w section_code
##  1:      B1             0            0
##  2:      B1             0            0
##  3:      B1             0            0
##  4:      B1             0            0
##  5:      B1             0            0
##  6:      B1             0            0
##  7:      B1             0            0
##  8:      B1             0            0
##  9:      B1             0            0
## 10:      B1             0            0
## 11:      B1             0            0
## 12:      B1             0            0
## 13:      B1             0            0
## 14:      B1             0            0
## 15:      B1             0            0
## 16:      B1             0            0
## 17:      B1             0            0
## 18:      B1             0            0
## 19:      B1             0            0
## 20:      B1             0            0
## 21:      B1             0            0
## 22:      B1             0            0
## 23:      B1             0            0
## 24:      B1             0            0
## 25:      B1             0            0
## 26:      B1             0            0
## 27:      B1             0            0
## 28:      B1             0            0
## 29:      B1             0            0
## 30:      B1             0            0
## 31:      B1             0            0
## 32:      B1             0            0
## 33:      B1             0            0
## 34:      B1             0            0
## 35:      A1             0            1
## 36:      A1             0            1
## 37:      A1             0            1
## 38:      A1             0            1
## 39:      A1             0            1
## 40:      A1             0            1
## 41:      A1             0            1
## 42:      A1             0            1
## 43:      A1             0            1
## 44:      A1             0            1
## 45:      A1             0            1
## 46:      A1             0            1
## 47:      A1             0            1
## 48:      A1             0            1
## 49:      A1             0            1
## 50:      A1             0            1
## 51:      A1             0            1
## 52:      A1             0            1
## 53:      A1             0            1
## 54:      A1             0            1
## 55:      A1             0            1
## 56:      A1             0            1
## 57:      A1             0            1
## 58:      A1             0            1
## 59:      A1             0            1
## 60:      A1             0            1
## 61:      A1             0            1
## 62:      A1             0            1
## 63:      A1             0            1
## 64:      A1             0            1
## 65:      A1             0            1
## 66:      A1             0            1
## 67:      A1             0            1
## 68:      A1             0            1
## 69:      A1             1            1
## 70:      A1             0            1
## 71:      A1             0            1
## 72:      A1             0            1
## 73:      A1             1            1
## 74:      A1             0            1
##     section starts_with_w section_code
\end{verbatim}

\hypertarget{section-3}{%
\paragraph{1.4}\label{section-3}}

Calculate the share of students in each section and treatment group In
order to do this, we will use aggregation features of data.table. This
is like SQL, if you've used it before. In a data.table, we can group by
variables (after the second comma) and count them. The code below counts
the students by college.

\begin{Shaded}
\begin{Highlighting}[]
\CommentTok{\# list(num\_students = .N) creates a variable called \textquotesingle{}num\_students\textquotesingle{} that counts students by college}
\NormalTok{agg\_data }\OtherTok{\textless{}{-}}\NormalTok{ class\_data[, }\FunctionTok{list}\NormalTok{(}\AttributeTok{num\_students =}\NormalTok{ .N), }\FunctionTok{list}\NormalTok{(section)]}
\NormalTok{agg\_data}
\end{Highlighting}
\end{Shaded}

\begin{verbatim}
##    section num_students
## 1:      B1           34
## 2:      A1           40
\end{verbatim}

Note, we now have two datasets, the original dataset `class\_data' and
the aggregate data `agg\_data'. We can then calculate the share of
students by major by dividing by the total number of students. We first
calculate the total number of students across major by using the
function `sum' to sum the values of the column `num\_students'. We then
create a column called `share\_students' by dividing num students by the
total number of students.

\begin{Shaded}
\begin{Highlighting}[]
\NormalTok{tot\_num\_students }\OtherTok{\textless{}{-}} \FunctionTok{sum}\NormalTok{(agg\_data[, num\_students])}
\NormalTok{agg\_data[, share\_students }\SpecialCharTok{:}\ErrorTok{=}\NormalTok{ num\_students}\SpecialCharTok{/}\NormalTok{tot\_num\_students]}
\NormalTok{agg\_data}
\end{Highlighting}
\end{Shaded}

\begin{verbatim}
##    section num_students share_students
## 1:      B1           34      0.4594595
## 2:      A1           40      0.5405405
\end{verbatim}

Below, repeat the above steps to calculate the share of students by each
value of the column `treatment\_group'.

\begin{Shaded}
\begin{Highlighting}[]
\CommentTok{\# Your code here:}
\NormalTok{treatment\_agg\_data }\OtherTok{\textless{}{-}}\NormalTok{ class\_data[, }\FunctionTok{list}\NormalTok{(}\AttributeTok{num\_students =}\NormalTok{ .N), }\FunctionTok{list}\NormalTok{(treatment\_group)]}
\NormalTok{treatment\_agg\_data[, share\_students }\SpecialCharTok{:}\ErrorTok{=}\NormalTok{ num\_students}\SpecialCharTok{/}\NormalTok{tot\_num\_students]}
\NormalTok{treatment\_agg\_data}
\end{Highlighting}
\end{Shaded}

\begin{verbatim}
##    treatment_group num_students share_students
## 1:               2           16      0.2162162
## 2:               4            9      0.1216216
## 3:               1           19      0.2567568
## 4:               0           18      0.2432432
## 5:               3           12      0.1621622
\end{verbatim}

\hypertarget{calculate-the-share-of-students-that-has-each-treatment-group-10-points}{%
\subsubsection{1.5 Calculate the share of students that has each
treatment group (10
points)}\label{calculate-the-share-of-students-that-has-each-treatment-group-10-points}}

We can now plot the shares. We do so by using the package `ggplot2'. We
can load this package by using the command library as below. Remember,
you must tell R to load packages.

\begin{Shaded}
\begin{Highlighting}[]
\FunctionTok{library}\NormalTok{(ggplot2)}
\end{Highlighting}
\end{Shaded}

Now, let's create a bar plot. The ggplot function takes in a dataset
(the first part of the function), and the values you are going to plot
(x is the section which will be on the x axis, y is the share\_students
and will be on the y axis). We then add the plot type: `geom\_bar(stat =
'identity')'

\begin{Shaded}
\begin{Highlighting}[]
\NormalTok{this\_plot }\OtherTok{\textless{}{-}} \FunctionTok{ggplot}\NormalTok{(agg\_data, }\FunctionTok{aes}\NormalTok{(}\AttributeTok{x =}\NormalTok{ section, }\AttributeTok{y =}\NormalTok{ share\_students)) }\SpecialCharTok{+} \FunctionTok{geom\_bar}\NormalTok{(}\AttributeTok{stat =} \StringTok{\textquotesingle{}identity\textquotesingle{}}\NormalTok{)}
\NormalTok{this\_plot}
\end{Highlighting}
\end{Shaded}

\includegraphics{assignment1_notebook_files/figure-latex/unnamed-chunk-14-1.pdf}

Repeat the above, but plot the share\_students by treatment group

\hypertarget{plot-a-histogram-of-the-treatment-groups-using-the-ggplot-function.-10-points}{%
\subsubsection{1.6 Plot a histogram of the treatment groups using the
ggplot function. (10
points)}\label{plot-a-histogram-of-the-treatment-groups-using-the-ggplot-function.-10-points}}

\begin{Shaded}
\begin{Highlighting}[]
\NormalTok{that\_plot }\OtherTok{\textless{}{-}} \FunctionTok{ggplot}\NormalTok{(treatment\_agg\_data, }\FunctionTok{aes}\NormalTok{(}\AttributeTok{x =}\NormalTok{ treatment\_group, }\AttributeTok{y =}\NormalTok{ share\_students)) }\SpecialCharTok{+} \FunctionTok{geom\_bar}\NormalTok{(}\AttributeTok{stat =} \StringTok{\textquotesingle{}identity\textquotesingle{}}\NormalTok{)}
\NormalTok{that\_plot}
\end{Highlighting}
\end{Shaded}

\includegraphics{assignment1_notebook_files/figure-latex/unnamed-chunk-16-1.pdf}

\hypertarget{randomly-assign-to-treatment-using-simple-randomization}{%
\subsubsection{1.7 Randomly assign to treatment using simple
randomization:}\label{randomly-assign-to-treatment-using-simple-randomization}}

Simple randomization amounts to flipping a coin separately for each
person to determine who gets treated. Let's flip a coin so that 40\% of
the students are in a new treatment group. To do so, we'll use the
rbernoulli function. Use ?rbernoulli to look at how it works.

\begin{Shaded}
\begin{Highlighting}[]
\CommentTok{\# 5 coin flips with probability 20\% of TRUE.}
\FunctionTok{rbernoulli}\NormalTok{(}\DecValTok{5}\NormalTok{, .}\DecValTok{20}\NormalTok{)}
\end{Highlighting}
\end{Shaded}

\begin{verbatim}
## [1] FALSE FALSE  TRUE FALSE  TRUE
\end{verbatim}

\begin{Shaded}
\begin{Highlighting}[]
\CommentTok{\# Let\textquotesingle{}s check that the mean is close to 20\%. Here we use the function mean to take the average of the sample.}
\FunctionTok{mean}\NormalTok{(}\FunctionTok{rbernoulli}\NormalTok{(}\DecValTok{1000}\NormalTok{, .}\DecValTok{20}\NormalTok{))}
\end{Highlighting}
\end{Shaded}

\begin{verbatim}
## [1] 0.207
\end{verbatim}

We now want to create a column in our dataset that has a 40\% of people
in the treatment. To do so, we need to tell rbernoulli how many times to
flip the coin (it's the number of people in the dataset) and the
probablity. Write the correct code based on the column below:

\begin{Shaded}
\begin{Highlighting}[]
\CommentTok{\# class\_data[, treatment := rbernoulli(number of people goes here, probability goes here)]}
\CommentTok{\# Your code here:}
\NormalTok{class\_data[, treatment }\SpecialCharTok{:}\ErrorTok{=} \FunctionTok{rbernoulli}\NormalTok{(tot\_num\_students, .}\DecValTok{4}\NormalTok{)]}
\end{Highlighting}
\end{Shaded}

Finally, let's look at the share of people in section A1 in the
treatment group. What do you get?

\begin{Shaded}
\begin{Highlighting}[]
\NormalTok{class\_data[treatment }\SpecialCharTok{==} \DecValTok{1}\NormalTok{, }\FunctionTok{mean}\NormalTok{(section }\SpecialCharTok{==} \StringTok{\textquotesingle{}A1\textquotesingle{}}\NormalTok{)]}
\end{Highlighting}
\end{Shaded}

\begin{verbatim}
## [1] 0.4333333
\end{verbatim}

What about the share of students in section B1 in the control group?

\begin{Shaded}
\begin{Highlighting}[]
\CommentTok{\# Your code here:}
\NormalTok{class\_data[treatment }\SpecialCharTok{==} \DecValTok{1}\NormalTok{, }\FunctionTok{mean}\NormalTok{(section }\SpecialCharTok{==} \StringTok{\textquotesingle{}B1\textquotesingle{}}\NormalTok{)]}
\end{Highlighting}
\end{Shaded}

\begin{verbatim}
## [1] 0.5666667
\end{verbatim}

\hypertarget{bonus}{%
\section{1.8 BONUS}\label{bonus}}

\begin{Shaded}
\begin{Highlighting}[]
\NormalTok{these\_plot }\OtherTok{\textless{}{-}} \FunctionTok{ggplot}\NormalTok{(agg\_data, }\FunctionTok{aes}\NormalTok{(}\AttributeTok{x =}\NormalTok{ section, }\AttributeTok{y =}\NormalTok{ share\_students, }\AttributeTok{fill =}\NormalTok{ section)) }\SpecialCharTok{+} 
  \FunctionTok{geom\_bar}\NormalTok{(}\AttributeTok{stat =} \StringTok{\textquotesingle{}identity\textquotesingle{}}\NormalTok{) }\SpecialCharTok{+}
  \FunctionTok{labs}\NormalTok{ (}\AttributeTok{x=} \StringTok{"Section"}\NormalTok{, }\AttributeTok{y=} \StringTok{"Share of Students"}\NormalTok{, }\AttributeTok{title=} \StringTok{"Section v Student Share"}\NormalTok{,}\AttributeTok{colour =} \StringTok{"section"}\NormalTok{, }\AttributeTok{subtitle =} \StringTok{"These Plots"}\NormalTok{)}

\NormalTok{these\_plot}
\end{Highlighting}
\end{Shaded}

\includegraphics{assignment1_notebook_files/figure-latex/unnamed-chunk-21-1.pdf}
\#\#\# Problem 2

\begin{Shaded}
\begin{Highlighting}[]
\CommentTok{\# R code if needed}
\NormalTok{PO }\OtherTok{\textless{}{-}} \FunctionTok{data.table}\NormalTok{(}\StringTok{"Users"} \OtherTok{=} \FunctionTok{seq}\NormalTok{(}\DecValTok{1}\NormalTok{,}\DecValTok{10}\NormalTok{))}
\NormalTok{PO[,}\StringTok{"Yes\_QA"}\SpecialCharTok{:}\ErrorTok{=} \FunctionTok{c}\NormalTok{(}\DecValTok{1100}\NormalTok{,}\DecValTok{100}\NormalTok{,}\DecValTok{500}\NormalTok{,}\DecValTok{900}\NormalTok{,}\DecValTok{1600}\NormalTok{,}\DecValTok{2000}\NormalTok{,}\DecValTok{1200}\NormalTok{,}\DecValTok{700}\NormalTok{,}\DecValTok{1100}\NormalTok{,}\DecValTok{140}\NormalTok{)]}
\NormalTok{PO[,}\StringTok{"No\_QA"} \SpecialCharTok{:}\ErrorTok{=} \FunctionTok{c}\NormalTok{(}\DecValTok{1100}\NormalTok{,}\DecValTok{600}\NormalTok{,}\DecValTok{500}\NormalTok{,}\DecValTok{900}\NormalTok{,}\DecValTok{700}\NormalTok{,}\DecValTok{2000}\NormalTok{,}\DecValTok{1200}\NormalTok{,}\DecValTok{700}\NormalTok{,}\DecValTok{1000}\NormalTok{,}\DecValTok{140}\NormalTok{)]}
\NormalTok{PO}\SpecialCharTok{$}\NormalTok{TE }\OtherTok{\textless{}{-}}\NormalTok{ PO}\SpecialCharTok{$}\NormalTok{Yes\_QA }\SpecialCharTok{{-}}\NormalTok{PO}\SpecialCharTok{$}\NormalTok{No\_QA}
\NormalTok{PO}
\end{Highlighting}
\end{Shaded}

\begin{verbatim}
##     Users Yes_QA No_QA   TE
##  1:     1   1100  1100    0
##  2:     2    100   600 -500
##  3:     3    500   500    0
##  4:     4    900   900    0
##  5:     5   1600   700  900
##  6:     6   2000  2000    0
##  7:     7   1200  1200    0
##  8:     8    700   700    0
##  9:     9   1100  1000  100
## 10:    10    140   140    0
\end{verbatim}

\hypertarget{question-a}{%
\subsection{Question a}\label{question-a}}

The treatment for this situation is a User saw the Q\&A page on a
Wayfair's furniture page. The outcomes are the reveune they would spend
if they had the treatment (User Saw Q\&A) or if they didn't have the
treatment (User did not see Q\&A).

If User 2 recieved the treatment (Saw Q\&A) then they would spend 100 if
USer 2 did not recieve the treatment (Did not See Q\&A) then they would
spend 600.

\hypertarget{question-b}{%
\subsection{Question b}\label{question-b}}

\begin{Shaded}
\begin{Highlighting}[]
\NormalTok{PO[,}\FunctionTok{list}\NormalTok{(Users,TE)]}
\end{Highlighting}
\end{Shaded}

\begin{verbatim}
##     Users   TE
##  1:     1    0
##  2:     2 -500
##  3:     3    0
##  4:     4    0
##  5:     5  900
##  6:     6    0
##  7:     7    0
##  8:     8    0
##  9:     9  100
## 10:    10    0
\end{verbatim}

Based on the data, Users 1,3,4,6,7,8,and 10 had no treatment effect.
Users 2 had a negative treatment effect of -500. Users 5 and 9 had a
positive treatment effect of 900 and 100 respectively.

\hypertarget{question-c}{%
\subsection{Question c}\label{question-c}}

User with 0 treatment effect were not influenced positvely or negatively
by seeing the Q\&A page. These users could have no questions or very set
on purchasing what they purchases. Users with a positive treatment
effect could have so an answer to a question they had and been swayed to
buy a product they were on the edge about. Users with a negative
treatment effect could have either not had seen an answer to their
question or seen an answers. Ultimately they were swayed to buying less
products than if they hadn't seen the Q\&A page.

\hypertarget{question-d}{%
\subsection{Question d}\label{question-d}}

\begin{Shaded}
\begin{Highlighting}[]
\NormalTok{avg }\OtherTok{\textless{}{-}} \FunctionTok{c}\NormalTok{(}\FunctionTok{mean}\NormalTok{(PO}\SpecialCharTok{$}\NormalTok{Yes\_QA),}\FunctionTok{mean}\NormalTok{(PO}\SpecialCharTok{$}\NormalTok{No\_QA),}\FunctionTok{mean}\NormalTok{(PO}\SpecialCharTok{$}\NormalTok{TE))}
\NormalTok{avg}
\end{Highlighting}
\end{Shaded}

\begin{verbatim}
## [1] 934 884  50
\end{verbatim}

The observed average for the treatment was 934. The observed average for
no treatment was 884. The impact of the treatment was 50. Due to seeing
the Q\&A page we can see an increase in 50.

\hypertarget{how-long-did-this-assignment-take-you-to-do-hours-how-hard-was-it-easy-reasonable-hard-too-hard}{%
\section{How long did this assignment take you to do (hours)? How hard
was it (easy, reasonable, hard, too
hard)?}\label{how-long-did-this-assignment-take-you-to-do-hours-how-hard-was-it-easy-reasonable-hard-too-hard}}

\end{document}
